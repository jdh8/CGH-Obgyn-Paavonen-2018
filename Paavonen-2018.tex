\documentclass{beamer}
\usepackage[no-math]{fontspec}
\usepackage{xeCJK}
\setCJKmainfont{Source Han Sans TW}
\hypersetup{colorlinks,linkcolor=}

\usetheme{CambridgeUS}
\title[Vaginitis]{Bacterial vaginosis and desquamative inflammatory vaginitis}
\subtitle{Paavonen \& Brunham, \textit{N Engl J Med}, 2018}
\author[Chen-Pang He]{何震邦 (Chen-Pang He), Intern}
\date{January 23, 2019}
\institute[CGH]{Cathay General Hospital}

\newcommand*{\solo}[1]{\centering\includegraphics[width=\textwidth, height=0.8\textheight, keepaspectratio]{#1}}

\begin{document}
\maketitle

\begin{frame}{Overview}
    \begin{itemize}
        \item Vaginal symptoms are remarkably common.
        \item Vaginal infections affect a woman's quality of life by causing
              frustration, anxiety, sexual dysfunction, and vulvovaginal
              discomfort.
        \item An abnormal vaginal microbiome has been linked to adverse
              pregnancy outcomes, pelvic inflammatory disease, an increased
              risk of sexually transmitted infections, and other reproductive
              health problems, such as a poor outcome of in vitro fertilization
              (IVF).
    \end{itemize}
\end{frame}

\begin{frame}{Topics}
    \begin{itemize}
        \item This review focuses on bacterial vaginosis and desquamative inflammatory vaginitis.
            \begin{itemize}
                \item Both are common, underrecognized disorders, and important new data about them have emerged.
            \end{itemize}
        \item Not discussed are trichomoniasis and vulvovaginal candidiasis.
    \end{itemize}
\end{frame}

\section{Vaginal microbiome}
\begin{frame}{Vaginal microbiome}
    \begin{itemize}
        \item Natural fluctuations occur during the reproductive cycle and
              throughout a woman's life.
        \item Principally influenced by the effects of estrogen on vaginal
              epithelial cells, the predominance of lactobacilli, and low pH
        \item Transiently influenced by use of antimicrobial agents, sexual
              activity, and menses
    \end{itemize}
\end{frame}

\begin{frame}{Table 1}
    \solo{T1.eps}
\end{frame}

\begin{frame}{Community state types}
    \begin{itemize}
        \item \textit{Lactobacillus crispatus}, \textit{L. gasseri}, and
              \textit{L. jensenii} usually occur as a single or predominant
              microorganism.
        \item \textit{L. iners} commonly occurs as a component of a
              polymicrobial flora, often transitioning to bacterial vaginosis.
        \item \textit{L. crispatus} excludes other organisms through low pH due
              to robust lactic acid production together with hydrogen peroxide
              and defensins.
    \end{itemize}
\end{frame}

\begin{frame}{Protective mechanisms}
    \begin{itemize}
        \item The presence of hydrogen peroxide--producing lactobacilli is
              associated with reduced levels of vaginal proinflammatory
              cytokines.
        \item A low-pH environment markedly inhibits bacterial growth.
    \end{itemize}
\end{frame}

\section{Bacterial vaginosis}
\begin{frame}{Bacterial vaginosis}
    \begin{itemize}
        \item A link between \textit{Gardnerella vaginalis} and abnormal
              vaginal discharge was first described in 1955.
        \item The syndrome was renamed non-specific vaginitis or anaerobic
              vaginosis because anaerobic organisms, in addition to
              \textit{G. vaginalis}, were observed.
        \item Bacterial vaginosis is a polymicrobial disorder of the vaginal
              microbiome that is characterized by the absence of vaginal
              lactobacilli.
    \end{itemize}
\end{frame}

\begin{frame}{Healthy cervicovaginal mucosa}
    \solo{F1a.jpg}
\end{frame}

\begin{frame}{Bacterial vaginosis}
    \solo{F1c.jpg}
\end{frame}

\begin{frame}{Desquamative inflammatory vaginitis}
    \solo{F1e.jpg}
\end{frame}

\section{Desquamative inflammatory vaginitis}

\begin{frame}{Table 2}
    \solo{T2.eps}
\end{frame}

\begin{frame}{Table 3}
    \solo{T3.eps}
\end{frame}

\end{document}
