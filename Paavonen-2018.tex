\documentclass{beamer}
\usepackage[no-math]{fontspec}
\usepackage{xeCJK}
\setCJKmainfont{Source Han Sans TW}
\hypersetup{colorlinks,linkcolor=}

\usetheme{CambridgeUS}
\title[Vaginitis]{Bacterial vaginosis and desquamative inflammatory vaginitis}
\subtitle{\textit{N Engl J Med} 379;23.  December 6, 2018}
\author[何震邦]{
    Jorma Paavonen\inst{1}, Robert C. Brunham\inst{2} \\[1ex]
    Speaker: Intern 何震邦 \\
    Advisor: Dr.~張美玲
}
\institute[CGH]{
    \inst{1} Helsinki University Hospital \\
    \inst{2} University of British Columbia \\[1ex]
    Cathay General Hospital
}
\date{January 23, 2019}

\newcommand*{\solo}[1]{\centering\includegraphics[width=\textwidth, height=0.8\textheight, keepaspectratio]{#1}}

\begin{document}
\maketitle

\begin{frame}{Overview}
    \begin{itemize}
        \item Vaginal symptoms are remarkably common.
        \item Vaginal infections affect a woman's quality of life by causing
              frustration, anxiety, sexual dysfunction, and vulvovaginal
              discomfort.
        \item An abnormal vaginal microbiome has been linked to adverse
              pregnancy outcomes, pelvic inflammatory disease, an increased
              risk of sexually transmitted infections, and other reproductive
              health problems, such as a poor outcome of in vitro fertilization
              (IVF).
    \end{itemize}
\end{frame}

\begin{frame}{Topics}
    \begin{itemize}
        \item This review focuses on bacterial vaginosis and desquamative inflammatory vaginitis.
            \begin{itemize}
                \item Both are common, underrecognized disorders, and important new data about them have emerged.
            \end{itemize}
        \item Not discussed are trichomoniasis and vulvovaginal candidiasis.
    \end{itemize}
\end{frame}

\section{Vaginal microbiome}
\begin{frame}{Vaginal microbiome}
    \begin{itemize}
        \item Natural fluctuations occur during the reproductive cycle and
              throughout a woman's life.
        \item Principally influenced by the effects of estrogen on vaginal
              epithelial cells, the predominance of lactobacilli, and low pH
        \item Transiently influenced by use of antimicrobial agents, sexual
              activity, and menses
    \end{itemize}
\end{frame}

\begin{frame}{Table 1}
    \solo{T1.eps}
\end{frame}

\begin{frame}{Community state types}
    \begin{itemize}
        \item \textit{Lactobacillus crispatus}, \textit{L. gasseri}, and
              \textit{L. jensenii} usually occur as a single or predominant
              microorganism.
        \item \textit{L. iners} commonly occurs as a component of a
              polymicrobial flora, often transitioning to bacterial vaginosis.
        \item \textit{L. crispatus} excludes other organisms through low pH due
              to robust lactic acid production together with hydrogen peroxide
              and defensins.
    \end{itemize}
\end{frame}

\begin{frame}{Protective mechanisms}
    \begin{itemize}
        \item The presence of hydrogen peroxide--producing lactobacilli is
              associated with reduced levels of vaginal proinflammatory
              cytokines.
        \item A low-pH environment markedly inhibits bacterial growth.
    \end{itemize}
\end{frame}

\section{Bacterial vaginosis}
\begin{frame}{Bacterial vaginosis}
    \begin{itemize}
        \item A link between \textit{Gardnerella vaginalis} and abnormal
              vaginal discharge was first described in 1955.
        \item The syndrome was renamed non-specific vaginitis or anaerobic
              vaginosis because anaerobic organisms, in addition to
              \textit{G. vaginalis}, were observed.
        \item Bacterial vaginosis is a polymicrobial disorder of the vaginal
              microbiome that is characterized by the absence of vaginal
              lactobacilli.
    \end{itemize}
\end{frame}

\begin{frame}{Epidemiology}
    \begin{itemize}
        \item One of the most common vaginal ecosystem--related microbiologic
              syndromes among women of childbearing age
        \item Incidence: 7.4 million cases each year in the United States
        \item Prevalence
            \begin{itemize}
                \item 15\% among pregnant women
                \item 20--25\% at student health clinics
                \item 30--40\% at sexually transmitted disease clinics
            \end{itemize}
    \end{itemize}
\end{frame}

\begin{frame}{Ethnic and geographic differences}
    \begin{itemize}
        \item The prevalence rates vary strikingly among ethnic groups and countries.
        \item Rates are generally higher in black and Hispanic populations and lower in white and Asian populations.
        \item Reasons are unknown.
    \end{itemize}
\end{frame}

\begin{frame}{Healthy cervicovaginal mucosa}
    \solo{F1a.jpg}
\end{frame}

\begin{frame}{Bacterial vaginosis}
    \solo{F1c.jpg}
\end{frame}

\begin{frame}{Desquamative inflammatory vaginitis}
    \solo{F1e.jpg}
\end{frame}

\begin{frame}{Characteristics}
    \begin{itemize}
        \item A milky, homogeneous, malodorous vaginal discharge that causes
              vulvovaginal discomfort and vulvar irritation
        \item Absence of clinically significant vaginal inflammation as
              indicated by an absence of neutrophils
        \item Fishy smell caused by the release of organic acids or polyamines
              on alkalinization of vaginal fluid, which are by-products of
              anaerobic bacterial metabolism
        \begin{itemize}
            \item The polymicrobial load is increased by a factor of up to
                  1000, as compared with normal flora.
        \end{itemize}
    \end{itemize}
\end{frame}

\begin{frame}{Transmission}
    \begin{itemize}
        \item The absence of a clear disease counterpart in males has made it
              difficult to determine whether bacterial vaginosis is sexually
              transmitted.
        \item A systematic review of randomized trials of treatment for male
              sexual partners to prevent recurrent bacterial vaginosis in women
              showed that none of the trials had sufficient power to determine
              the role of the male partner in the recurrence of bacterial
              vaginosis.
        \item Another review concluded that, as compared with placebo,
              antibiotic treatment for the sexual partners of women treated for
              bacterial vaginosis had no effect on rates of clinical or
              symptomatic improvement among the women or on the rate of
              recurrence of bacterial vaginosis for up to 12 weeks after
              treatment.
    \end{itemize}
\end{frame}

\begin{frame}{Similarities to sexually transmitted diseases}
    \begin{itemize}
        \item However, bacterial vaginosis and sexually transmitted infections
              have many characteristics in common, and several findings are
              consistent with a strong association between incident bacterial
              vaginosis and sexual activity.
        \item Thus, there may be either unmeasured confounders in these studies
              or a transmissible microbial component of bacterial vaginosis
              that has not yet been identified.
    \end{itemize}
\end{frame}

\section{Desquamative inflammatory vaginitis}

\begin{frame}{Table 2}
    \solo{T2.eps}
\end{frame}

\begin{frame}{Table 3}
    \solo{T3.eps}
\end{frame}

\end{document}
